\documentclass{article}
\input{defs}

\errlang{unknown-identifier}
\newcommand{\evalz}{\mathit{eval_{\mathbf{Z}}}}
\newcommand{\pevalz}{\mathit{PE_{\mathbf{Z}}}}
\newcommand{\Resid}{\oblset{Residual}}
\newcommand{\ResidFrame}{\oblset{ResidualFrame}}
\newcommand{\IVars}{\oblset{IVars}}
\newcommand{\Input}{\mathit{input}}

\title{Z-Language}
\author{Christian, Pavel, Pauline}

\begin{document}

\maketitle

%  \centering
\noindent\textbf{Program Static Syntax}\\ \\
\begin{math}
  \begin{array}{rcl}
    \multicolumn{3}{l}{
      t \in \Term,\quad
      n \in \Z,\quad
      x \in \Var, \quad
    } \\
    t & ::= & n | x | \<if> t \<then> t \<else> t | t ++ t\\
  \end{array}
\end{math}
\\

\noindent\textbf{Program Runtime Syntax}
%
\begin{align*}
  &\Gbox{nr \in \Resid \subseteq \Term} \\
  &F \in \Frame,\quad \Gbox{R \in \ResidFrame},\quad o \in \Obs \\
  &\Gbox{i \in \IVars = \Var \rightharpoonup \Z}
\end{align*}

\begin{displaymath}
  \begin{array}{rcl}
    R & ::= & nr + [] | \<if> nr \<then> nr \<else> [] \\
    nr & ::= & n | x | R[nr] \\
    F & ::= & [] | \Gbox{R} | \<if> [] \<then> t \<else> t |
    \Gbox{\<if> nr \<then> [] \<else> t} | [] + t \\
    o & ::= & n | \<unknown-identifier>
  \end{array}
\end{displaymath}

\vspace{0.5cm}

$\boxed{{-->} \subseteq \Term \times \Term}$
\quad\textbf{Single-step Reduction}\\
\begin{mathpar}
  \inference[(splus)]{}{n_1 ++ n_2 --> n_1 + n_2} \and
  \inference[(sif-t)]{n \neq 0}{\<if> n \<then> t_2 \<else> t_3 --> t_2} \and
  \inference[(sif-f)]{}{\<if> 0 \<then> t_2 \<else> t_3 --> t_3} \and
  \inference[(sframe)]{p --> p'}{F[p] --> F[p']}
\end{mathpar}

\vspace{0.5cm}

\boxed{\Input : \Term \times \IVars -> \Term} \\
\begin{displaymath}
\begin{array}{l}
    \Input(n, i) = n \\
    \Input(x, i) = n,\, \exists n \in \Z.\, i(x) = n \\
    \Input(x, i) = x,\, \nexists n \in \Z.\, i(x) = n \\
    \Input(\<if> t_1 \<then> t_2 \<else> t_3, i) =
    \<if> \Input(t_1, i) \<then> \Input(t_2, i) \<else> \Input(t_3, i) \\
    \Input(t_1 ++ t_2, i) = \Input(t_1, i) ++ \Input(t_2, i) \\
\end{array}
\end{displaymath}

\boxed{\pevalz : \Term \times \IVars -> \Resid} \\
%
\begin{displaymath}
\begin{array}{l}
\pevalz(t, i) = nr  \;\text{if}\; input(t, i) -->* nr \\
\end{array}
\end{displaymath}

\boxed{\evalz : \Term \times \IVars  -> \Obs} \\
%
\begin{displaymath}
\begin{array}{l}
\evalz(t, i) = \<unknown-identifier>  \;\text{if}\; \Vars(\Input(t, i)) \neq \emptyset \\
\evalz(t, i) = n  \;\text{if}\; input(t, i) -->* n \\
\end{array}
\end{displaymath}

\newpage

Let: $t = \<if> \x ++ \y \<then> \x ++ 3 \<else> \x ++ 5$

\vspace{0.5cm}

Calculate $\evalz(t, i)$ with $i = \set{\braket{\x, 3}, \braket{\y, 4}}$:
\begin{gather*}
  \Input(t, i) = \<if> 3 ++ 4 \<then> 3 ++ 3 \<else> 3 ++ 5 \\
  \<if> 3 ++ 4 \<then> 3 ++ 3 \<else> 3 ++ 5 -->
  \<if> 7 \<then> 3 ++ 3 \<else> 3 ++ 5 --> 3 ++ 3 --> 6 \\
  \implies \evalz(t, i) = 6
\end{gather*}


Now, calculate $\pevalz(t, i')$ with $i = \set{\braket{\x, 3}}$:
\begin{gather*}
  \Input(t, i') = \<if> 3 ++ y \<then> 3 ++ 3 \<else> 3 ++ 5 \\
  \<if> \underbrace{3 ++ y}_{nr} \<then> 3 ++ 3 \<else> 3 ++ 5 -->
  \<if> \underbrace{3 ++ y}_{nr} \<then>
  \underbrace{6}_{nr} \<else> 3 ++ 5 -->
  \<if> \underbrace{3 ++ y}_{nr} \<then>
  \underbrace{6}_{nr} \<else> \underbrace{8}_{nr} \\
  \implies \pevalz(t, i') = \<if> 3 ++ y \<then> 6 \<else> 8
\end{gather*}

\end{document}
